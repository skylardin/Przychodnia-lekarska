\documentclass[a4paper]{article} \usepackage{polski} \usepackage[cp1250]{inputenc} \usepackage{url}

\title{\bf{Elektroniczna przychodnia lekarska (PHP, MYSQL)}} \author{{\em Paweł Wilczek }{\em Łukasz Szkaradek } {\em Łukasz Mamak}} \date{}
\usepackage{hyperref}

\begin{document}

\begin{titlepage} \maketitle \thispagestyle{empty} \bigskip \begin{center} Zespołowe przedsięwzięcie inżynierskie

Informatyka

Rok. akad. 2017/2018, sem. I

Prowadzący: dr hab. Marcin Mazur \end{center} \end{titlepage}

\tableofcontents \thispagestyle{empty}


\newpage

\section{Opis projektu}


\subsection{Członkowie zespołu}

\begin{enumerate} \item Łukasz Szkaradek (kierownik projektu). \item Paweł Wilczek. \item Łukasz Mamak. \end{enumerate}

\subsection{Cel projektu (produkt)}


Celem projektu jest stworzenie strony internetowej z zastosowaniem php i mysql, która ma na celu obsługę wirtualnej przychodni lekarskiej oraz zwiększenie jej wydajności.

\subsection{Potencjalny odbiorca produktu (klient)}


Klientem może być przychodnia lekarska chcąca usprawnić wydajność niskim kosztem.

\subsection{Metodyka}


Projekt będzie realizowany przy użyciu (zaadaptowanej do istniejących warunków) metodyki {\em Scrum}.

\section{Wymagania użytkownika}


\subsection{User story 1}


Jako pacjent, chciałbym mieć możliwość zapisu do danego specjalisty za pośrednictwem strony internetowej przychodni, dzięki czemu nie musiałbym udać się tam osobiście i mógłbym przez to lepiej zarządzać czasem jaki mam do dyspozycji danego dnia.

\subsection{User story 2}

Jako pacjent chciałbym mieć wgląd do mojej karty pacjenta poprzez stronę przychodni tak abym mógł łatwo sprawdzić aktualną listę moich wizyt z lekarzem.

\subsection{User story 3}


Jako dyrektor placówki chciałbym posiadać konto admina by zarządzać elektroniczną przychodnią tak, aby w pełni czuwać nad moimi pracownikami.

\subsection{User story 4}


Jako lekarz chce żeby oprogramowanie przychodni miało łatwy i szybki dostęp do wszystkich moich aktualnych wizyt.


\subsection{User story 5}


Jako recepcjonista w przychodni chciałbym mieć dostęp do elektronicznej bazy przychodni co w przeciwieństwie do tradycyjnej formy skróciło by mój czas reakcji i zmniejszyło ilość mojej pracy.

\subsection{User story 6}

Jako pacjent chciałbym mieć dostęp do aktualności w przychodni takie jak opóźnienia w przyjmowaniu przez lekarzy, tak aby w takim przypadku inaczej zagospodarować czas.

\subsection{User story 7}


Jako dyrektor chciałbym, aby platforma internetowa zwiększyła wydajność naszej placówki poprzez łatwość w rejestracji i katalogowaniu danych.

\subsection{User story 8}


Jako lekarz chciałbym mieć dostęp do elektronicznej wersji karty pacjenta co umożliwiało by mi poznanie historii choroby pacjenta i jej edycje w razie konieczności.


\section{Harmonogram}

\subsection{Rejestr zadań (Product Backlog)}


\begin{itemize} \item Data rozpoczęcia: 11.10.2017. \item Data zakończenia: 15.11.2017. \end{itemize}

\subsection{Sprint 1}


\begin{itemize} \item Data rozpoczęcia: 15.11.2017. \item Data zakończenia: 29.11.2017. \item Scrum Master: Łukasz Szkaradek. \item Product Owner: Łukasz Mamak. \item Development Team: Łukasz Mamak, Paweł Wilczek, Łukasz Szkaradek. \end{itemize}

\subsection{Sprint 2}


\begin{itemize} \item Data rozpoczęcia: 29.11.2017. \item Data zakończenia: 20.12.2017. \item Scrum Master: Łukasz Mamak. \item Product Owner: Paweł Wilczek. \item Development Team: Paweł Wilczek, Łukasz Szkaradek, Łukasz Mamak. \end{itemize}

\subsection{Sprint 3}


\begin{itemize} \item Data rozpoczęcia: 20.12.2017. \item Data zakończenia: 10.01.2018. \item Scrum Master: Paweł Wilczek. \item Product Owner: Łukasz Szkaradek. \item Development Team: Łukasz Szkaradek, Łukasz Mamak, Paweł Wilczek. \end{itemize}


\begin{itemize} \item Data rozpoczęcia: 10.01.2018. \item Data zakończenia: 24.01.2018. \end{itemize}

\subsection{Sprint 4}


\section{Product Backlog}


\subsection{Backlog Item 1} \paragraph{Tytuł zadania.} Stworzenie  bazy danych. \paragraph{Opis zadania.} Stworzenie bazy danych, która służyć będzie do testowania początkowych zadań. \paragraph{Priorytet.} Średni. \paragraph{Definition of Done.} Stworzenie w MYSQL bazy danych i tablic: 
\newline

 \textbf{SPECJALIZACJE: }

\begin{itemize}
\item id specjalizacji (automatycznie przypisywany numer),
\item nazwa specjalizacji (nazwa specjalizacji), 
\end{itemize}


 \textbf{AKTUALNOŚCI: }

\begin{itemize}
\item id aktualnosci (automatycznie przypisywany numer),
\item data (data napisania aktualności) , 
\item opis (treść aktualności)
\end{itemize}

\textbf{PACJENCI: }
  
\begin{itemize}
\item id pacjenta (automatycznie przypisywany numer), 
\item imie (imię pacjenta), 
\item nazwisko (nazwisko pacjenta), 
\item pesel (pesel pacjenta), 
\item karta pacjenta (treść karty pacjenta), 
\item nr telefonu
\item kod (automatycznie generowany kod, który wraz z peselem będzie umożliwiał logowanie się pacjenta)
\end {itemize}\newline

 \textbf{SPOTKANIA:}
 
\begin{itemize}
\item id spotkania (automatycznie przypisywany numer),
\item id specjalizacji (numer odnoszący się do konkretnego wiersza w tablicy specjalizacje),
\item id lekarza (numer odnoszący się do konkretnego wiersza w tablicy lekarze),
\item id osoby (numer odnoszący się do konkretnego wiersza w tablicy pacjenci),
\item data odbycia (data wpisywana przez recepcjonistę, która określa datę spotkania z lekarzem),
\item data zapisu (automatycznie przypisywana data podczas tworzenia wiersza),
\item stan (równa się 0 lub 1, w zależności czy spotkanie zostało zaakceptowane przez recepcjonistę)
\end {itemize}\newline
\textbf{LEKARZE:}

\begin {itemize}
\item id lekarza (automatycznie przypisywany numer),
\item login (login używany przy logowaniu na stronie recepcji),
\item haslo (hasło używane przy logowaniu na stronie recepcji),
\item imie (imię lekarza),
\item nazwisko (nazwisko lekarza),
\item id specjalizacji (numer odnoszący się do konkretnego wiersza w tablicy specjalizacje, określa jaką specjalizację posiada dany lekarz),
\item nr pokoju (numer pokoju w którym dany lekarz przyjmuje),
 
 \end {itemize}


\subsection{Backlog Item 2}\paragraph{Tytuł zadania.} Strona tytułowa oraz formularz logowania pacjenta. \paragraph{Opis zadania.} Stworzenie strony głównej wraz z interfejsem i logowaniem dla pacjenta. \paragraph{Priorytet.} Średni. \paragraph{Definition of Done.} Napisanie w języku PHP strony głównej w której będą zawarte:
\newline
\newline 1. Formularz logowania pacjenta (poprzez pesel i kod).
\newline
\newline 2. Odnośniki do reszty podstron takich jak:
\begin{itemize}
\item formularz rejestracji pacjenta
\item strony recepcji.
\end {itemize}

\textit{\textbf{Strona zawierać będzie skrypt, który po wykryciu zmiennych wysłanych przez formularz logowania pacjenta wyświetli:}}

\begin{itemize}
\item imię pacjenta, 
\item nazwisko pacjenta, 
\item pesel pacjenta, 
\item listę umówionych spotkań z lekarzami,
\item formularz przez, który tworzyć będzie spotkania wraz z skryptem obsługującym daną operację (dodanie do tablicy "spotkania" danych: id specjalizacji, id osoby).

\end{itemize}

\subsection{Backlog Item 3}\paragraph{Tytuł zadania.} Aktualności. \paragraph{Opis zadania.} Dodanie do strony głównej listę aktualności. \paragraph{Priorytet.} Niski. \paragraph{Definition of Done.}  Edycja strony głównej tak aby w przejrzysty sposób wyświetlała dane z tablicy "aktualności":
- Data aktualności (data),
- Treść aktualności (opis).

\subsection{Backlog Item 4}\paragraph{Tytuł zadania.} Formularz zapisu dla pacjentów. \paragraph{Opis zadania.} Stworzenie strony dzięki której będzie można zapisać się do listy pacjentów. \paragraph{Priorytet.} Niski. \paragraph{Definition of Done.} Napisanie strony w języku PHP zawierającej formularz z danymi do wprowadzenia (Imię, Nazwisko, Pesel, Telefon kontaktowy). Po otrzymaniu danych zostaną one wprowadzone do tablicy "pacjenci". Ponadto strona po wypełnieniu formularza generować będzie 5-cyfrowy kod, który również będzie wprowadzany do tablicy.



\subsection{Backlog Item 5}\paragraph{Tytuł zadania.} Aktualizacja bazy danych. \paragraph{Opis zadania.} Dodanie do bazy danych potrzebnych tablic i ewentualna edycja starych tablic w razie potrzeby. \paragraph{Priorytet.} Wysoki. \paragraph{Definition of Done.} Dodanie do bazy danych, tablic:
\newline
\newline
 \textbf{RECEPCJA:}
 \begin{itemize}
\item id recepcjonisty  (automatycznie przypisywany numer),
\item imie (imię recepcjonisty),
\item nazwisko (nazwisko recepcjonisty),
\item login (login używany przy logowaniu na stronie recepcji),
\item  haslo (hasło używane przy logowaniu na stronie recepcji)
\end {itemize}
\newline
\textbf{DYREKTOR:}
\begin {itemize}
\item id dyrektora  (automatycznie przypisywany numer),
\item imie (imię dyrektora),
\item nazwisko (nazwisko dyrektora),
\item login (login używany przy logowaniu na stronie recepcji),
\item haslo (hasło używane przy logowaniu na stronie recepcji).
\end {itemize}
\newline
\textit{\textbf{Stworzenie 2 widoków, ułatwiających obsługę strony:}}
\newline
 
\textbf{LEKARZE:}
\begin {itemize}
\item imie (imię lekarza z tablicy "lekarze"),
\item nazwisko (nazwisko lekarza z tablicy "lekarze"),
\item specjalizacja (nazwa specjalizacji z tablicy "specjalizacje", odpowiednia dla konkretnego lekarza),
\item  nr pokoju (nr pokoju z tablicy "lekarze")
\end {itemize}\newline

\textbf{LISTA:}
\begin {itemize}
\item id spotkania (id spotkania z tablicy "spotkania"),
\item id osoby (id pacjenta z tablicy "pacjenci" odpowiednie dla konkretnego spotkania),
\item id lekarza (id lekarza z tablicy "lekarze" odpowiednie dla konkretnego spotkania),
\item pesel (pesel z tablicy "pacjenci"),
\item imie osoby (imię osoby z tablicy "pacjenci"),
\item nazwisko osoby (nazwisko osoby z tablicy "pacjenci"),
\item specjalizacja (nazwa specjalizacji z tablicy "specjalizacje"),
\item imie lekarza (imię lekarza z tablicy "lekarze"),
\item nazwisko lekarza (nazwisko lekarza z tablicy "lekarze"),
\item nr pokoju (numer pokoju z tablicy "lekarze"),
\item data odbycia (data odbycia z tablicy "spotkania"),
\item data zapisu (data zapisu z tablicy "spotkania"),
\item stan (stan z tablicy "spotkania")

\end {itemize}



\subsection{Backlog Item 6}\paragraph{Tytuł zadania.} Strona logowania pracowników. \paragraph{Opis zadania.} Napisanie strony, która będzie dawać dostęp do logowania i wyświetlania listy opcji dla: dyrektora, recepcjonisty i lekarza. \paragraph{Priorytet.} Średni. \paragraph{Definition of Done.} Stworzenie strony na, której będzie formularz z danymi do wprowadzenia: login, hasło. Wraz z skryptem który po ich otrzymaniu przeszuka tablice: lekarze, dyrektor, recepcja w poszukiwaniu prawidłowości. Jeżeli znajdzie wykorzysta funkcję "session" i wpiszę do niej dane takie jak imię, nazwisko, id oraz tablica z informacją z której tablicy dana osoba pochodzi. W przypadku wykrycia zmiennych w "session", wypisuje imię, nazwisko oraz id danej osoby, oraz w zależności od zmiennej tablica listę odnośników do strony "zarzadzaj" \\

 \textbf{DLA LEKARZE:}
 \begin {itemize}
\item odnośnik wraz z informacją w formie GET proszącą o listę spotkań.
\item odnośnik wraz z informacją w formie GET proszącą o kartę pacjenta.
\end {itemize} \\

\textbf{DLA RECEPCJA:}
\begin {itemize}
\item odnośnik wraz z informacją w formie GET proszącą o lista osób zapisanych. 
\item odnośnik wraz z informacją w formie GET proszącą o lista lekarzy .
\item odnośnik wraz z informacją w formie GET proszącą o lista dostępnych specjalizacji. 
\end{itemize}

 \textbf{DLA DYREKTOR:}
 \begin {itemize}
\item odnośniki wraz z informacjami w formie GET proszącą o możliwość edycji, usuwania, dodawania do tablic:
\item a) lekarze
\item b) specjalizacje
\item c) recepcja
\item d) spotkania
\item e) pacjenci
\item f) aktualnosci
\end {itemize}


\subsection{Backlog Item 7}\paragraph{Tytuł zadania.} Wyświetlanie, edycja, usuwanie. \paragraph{Opis zadania.} Stworzenie strony obsługujących edycję, wyświetlanie i usuwanie danych z tablic. \paragraph{Priorytet.} Wysoki. \paragraph{Definition of Done.} Stworzenie strony "zarzadzanie", która po otrzymaniu informacji formą GET i zweryfikowaniu uprawnień z funkcji "session", będzie udostępniać formie formularza wraz skryptem je obsługującym opcje dla osób z tablicy:\newline

\textbf{LEKARZE:}
\begin {itemize}
\item listę spotkań  (wyświetlanie),
\item kartę pacjenta (wyświetlanie i edycja)
\end {itemize}
\\
\textbf{RECEPCJA:}
\begin{itemize}
\item listę osób zapisanych (wyświetlanie i edycja),
\item listę lekarzy (wyświetlanie),
\item listę dostępnych specjalizacji (wyświetlanie)
\end {itemize}
\\
\textbf{DYREKTOR:}
\begin {itemize}
\item listę lekarze (wyświetlanie, edycja i usuwanie),
\item listę specjalizacje (wyświetlanie, edycja i usuwanie),
\item listę recepcja (wyświetlanie, edycja i usuwanie),
\item listę spotkania (wyświetlanie, edycja i usuwanie),
\item listę pacjenci (wyświetlanie, edycja i usuwanie),
\item listę aktualnosci (wyświetlanie, edycja i usuwanie)
\end {itemize}

\section{Sprint 1} \subsection{Cel} Stworzenie bazy danych dzięki której możliwe będzie przeglądanie strony tytułowej i formularza zapisu pacjentów. \subsection{Sprint Planning/Backlog}

\paragraph{Tytuł zadania.} Stworzenie początkowej bazy danych. \begin{itemize} \item Estymata: M. \end{itemize}
\paragraph{Tytuł zadania.}  Strona tytułowa oraz formularz logowania pacjenta. \begin{itemize} \item Estymata: L. \end{itemize}
\paragraph{Tytuł zadania.} Aktualności. \begin{itemize} \item Estymata: S. \end{itemize}



\subsection{Realizacja}


\paragraph{Tytuł zadania.} Stworzenie początkowej bazy danych. \subparagraph{Wykonawca.} Łukasz Mamak \subparagraph{Realizacja.} Zadaniem do wykonania było stworzenie niezbędnej do dalszych kroków bazy danych. Realizacja tegoż zadania przebiegła bez większych komplikacji. W skrypcie zostało utworzonych kilka tabel, począwszy od aktualności, poprzez tabele pacjenci, lekarze, na tabeli spotkania kończąc. W każdej tabeli zdefiniowany został klucz główny dla wartości "niepowtarzalnych" w zależności od tabeli. Zdefiniowane zostały także klucze obce, które zostały wykorzystane do utworzenia relacji między parą tabel. Podsumowując, zadanie zostało zrealizowane w całości, bez istotnych do odnotowania problemów, nie wychodzących poza ramy błędnego wpisania składni, lub też drobnych literówek. 
\newline
\newline
    Kod programu (środowisko \texttt{verbatim}): \begin{verbatim}

CREATE TABLE aktualnosci (
id_aktualnosci INT NOT NULL auto_increment,
data TIMESTAMP NOT NULL,
opis VARCHAR(50) NOT NULL,

CONSTRAINT c_pk0 PRIMARY KEY(id_aktualnosci)
) ENGINE = InnoDB; 

CREATE TABLE pacjenci (
id_pacjenta INT NOT NULL auto_increment,
imie VARCHAR(50) NOT NULL,
nazwisko VARCHAR(50) NOT NULL,
karta_pacjenta VARCHAR(255) NOT NULL,
pesel BIGINT NOT NULL,
nr_telefonu BIGINT NOT NULL,
kod VARCHAR(50) NOT NULL,

CONSTRAINT c_pk PRIMARY KEY(id_osoby)
) ENGINE = InnoDB; 

CREATE TABLE lekarze (
id_lekarza INT NOT NULL auto_increment,
login VARCHAR(50) NOT NULL,
haslo VARCHAR(50) NOT NULL,
imie VARCHAR(50) NOT NULL,
nazwisko VARCHAR(50) NOT NULL,
id_specjalizacji INT NOT NULL,
nr_pokoju VARCHAR(50) NOT NULL,

CONSTRAINT c_pk2 PRIMARY KEY(id_lekarza)
) ENGINE = InnoDB; 


CREATE TABLE spotkania (
id_spotkania INT NOT NULL auto_increment,
id_specjalizacji INT NOT NULL,
id_lekarza INT NOT NULL,
id_osoby INT NOT NULL,
data_odbycia VARCHAR(50),
data_zapisu TIMESTAMP NOT NULL,
stan enum('0','1') NOT NULL DEFAULT '0',


) ENGINE = InnoDB; 

\end{verbatim}.

\paragraph{Tytuł zadania.}  Strona tytułowa oraz formularz logowania pacjenta. \subparagraph{Wykonawca.} Łukasz Szkaradek \subparagraph{Realizacja.} Celem zadania było wykonanie strony internetowej w której zawarte będą logowanie pacjenta, wyświetlanie spotkań danego pacjenta oraz ich dodawanie poprzez formularz. Wykonanie zadania odbyło się bez większych trudności, choć wykonanie niektórych algorytmów takich jak wyświetlanie wyboru godzin z pominięciem już zarezerwowanych oraz przeszłych godzin czy wyświetlanie odpowiednich dat dla wybranych dni wymagało większego wysiłku. Przydatna przy wykonaniu zadania była funkcja session, która przechowuje dane aż do ich usunięcia lub wyłączenia przeglądarki. Przy przesyłaniu danych z formularza została użyta forma POST. Zadanie zostało wykonane w całości, a wszystkie jego funkcje działają prawidłowo.
\newline
\newline
Kod programu (środowisko \texttt{PHP}): \begin{verbatim} 
<?php
ob_start();
date_default_timezone_set('Europe/Warsaw');
session_start();
$host="<<ip serwera>>";
$db_user="<<nazwa użytkownika>>";
$db_password="<<hasło użytkowinka>>";
$database="<<nazwa bazy danych>>";
$link= mysqli_connect($host,$db_user,$db_password,$database);
$zat="call zatw()";
$zat_w=mysqli_query($link,$zat);
if (isset($_GET[powrot])) {
session_destroy();
header("Location: index.php");
}
if (isset($_POST[stan])){
$zapytanie="SELECT * FROM pacjenci WHERE pesel='$_POST[pesel]' and kod='$_POST[kod]'";
$wykonaj=mysqli_query($link,$zapytanie);
if(@mysqli_num_rows($wykonaj)){
while($wiersz=mysqli_fetch_assoc($wykonaj)) {
$_SESSION[zalogowany] = "pacjent";
$_SESSION[id_p] = $wiersz['id_pacjenta'];
$_SESSION['baza'] = 'pacjent';
}
} else {
$brak='1';
}
}
?>
<html>
<head>
<title>Przychodnia lekarska</title>
<META http-equiv=Content-Language content=pl>
<META http-equiv=Content-Type content="text/html; charset=windows-1250">
<style>
a {
    color: black;
    text-decoration: none;
}
</style>
</head>
\end{verbatim}.


\paragraph{Tytuł zadania.} Aktualności. \subparagraph{Wykonawca.} Paweł Wilczek \subparagraph{Realizacja.} Zadanie polegało na utworzeniu aktualności na stronie głównej przychodni lekarskiej dzięki wyłuskaniu odpowiednich informacji z tabeli "aktualności" i umieszczeniu ich na stronie głównej dzięki skryptowi PHP. 

Tabela aktualności:
\begin{verbatim}
CREATE TABLE aktualnosci (
id_aktualnosci INT NOT NULL auto_increment,
data TIMESTAMP NOT NULL,
opis VARCHAR(50) NOT NULL,

CONSTRAINT c_pk0 PRIMARY KEY(id_aktualnosci)
) ENGINE = InnoDB;
\end{verbatim}
Skrypt PHP:
\begin{verbatim}
echo '<br><br><br>
<table border="1px">
<tr><td style="text-align: center;" colspan="2">Aktualności</td></tr>';
$aktualnosci="SELECT * FROM aktualnosci";
$wykonaj=mysqli_query($link,$aktualnosci);
if(@mysqli_num_rows($wykonaj)){
	while($wiersz=mysqli_fetch_assoc($wykonaj)) {
		echo '<tr><td width="150">'.$wiersz['data'].'</td><td width="950">'.$wiersz['opis'].'</tr></table>';
}
}
}
\end{verbatim}

\subsection{Sprint Review/Demo} Naszym celem była realizacja 3 wcześniej ustalonych i zaplanowanych zadań. Dotyczyły one stworzenia bazy danych, strony tytułowej oraz formularza logowania pacjenta a także aktualności. Podczas realizacji wcześniej przydzielonych zadań, żaden z członków zespołu nie napotkał na istotne, konieczne do odnotowania problemy. Wszystkie z przewidzianych zadań zostały wykonane terminowo, co za tym idzie przyrost  został osiągnięty. Z racji tematyki naszego projektu, przyrost, postęp bedzie dostępny na stronie internetowej przychodni, w adresie poniżej.\newline
\newline
\textit{Demonstracja przyrostu produktu:}\newline
\newline
\textbf{Link do strony}: http://luki9696.000webhostapp.com/projekt/index.php
 
   

\section{Sprint 2}
\label{Sprint22}
\subsection{Cel}\label{Sprint2cel} Sprint ma na celu wykonanie dwóch zadań. Pierwszym zadaniem będzie uaktualnienie bazy danych o nowe tabele, wymagane do dalszego rozwoju strony, jak i stworzenie widoków, które ułatwiać będą wyświetlanie danych z tabeli. Natomiast drugim celem sprintu jest stworzenie podstrony, na której zawarty będzie formularz przez który będą mogli rejestrować się pacjenci, strona musi zawierać również kod, który będzie umożliwiał wykonanie operacji wpisania pacjenta do bazy.

\subsection{Sprint Planning/Backlog}
\label{Sprint2SPB}
\paragraph{Tytuł zadania.} Formularz zapisu dla pacjentów. \begin{itemize} \item Estymata: M. \end{itemize}

\paragraph{Tytuł zadania.} Aktualizacja bazy danych. \begin{itemize} \item Estymata: XL. \end{itemize}


\subsection{Realizacja}
\label{Realizacja2}
\paragraph{Tytuł zadania.}  Aktualizacja bazy danych. \subparagraph{Wykonawca.} Łukasz Szkaradek. \subparagraph{Realizacja.} Zadanie zostało wykonane bez problemów zostały utworzone 3 nowe tabele oraz 2 widoki ułatwiające wyświetlanie danych z tabeli. 
\newline
\newline
Kod programu (środowisko \texttt{verbatim}): \begin{verbatim} 

CREATE TABLE specjalizacje (
id_specjalizacji INT NOT NULL auto_increment,
nazwa_specjalizacji VARCHAR(50) NOT NULL,

CONSTRAINT c_pk1 PRIMARY KEY(id_specjalizacji)

) ENGINE = InnoDB; 

CREATE TABLE recepcja (
id_recepcjonisty INT NOT NULL auto_increment,
imie VARCHAR(50) NOT NULL,
nazwisko VARCHAR(50) NOT NULL,
login VARCHAR(50) NOT NULL,
haslo VARCHAR(50) NOT NULL,

CONSTRAINT c_pk2 PRIMARY KEY(id_recepcjonisty)

) ENGINE = InnoDB; 

CREATE TABLE dyrektor (
id_dyrektora INT NOT NULL auto_increment,
imie VARCHAR(50) NOT NULL,
nazwisko VARCHAR(50) NOT NULL,
login VARCHAR(50) NOT NULL,
haslo VARCHAR(50) NOT NULL,

CONSTRAINT c_pk3 PRIMARY KEY(id_dyrektora)

) ENGINE = InnoDB; 

CREATE OR REPLACE VIEW widok_lekarze(imie, nazwisko, specjalizacja, nr_pokoju) AS SELECT imie, nazwisko, specjalizacje.nazwa_specjalizacji, nr_pokoju FROM lekarze, specjalizacje WHERE lekarze.id_specjalizacji=specjalizacje.id_specjalizacji;
CREATE OR REPLACE VIEW widok_lista(id_spotkania, id_osoby, id_lekarza, pesel, imie_osoby, nazwisko_osoby, specjalizacja, imie_lekarza, nazwisko_lekarza, nr_pokoju, data_odbycia, data_zapisu, stan) AS SELECT id_spotkania, pacjenci.id_pacjenta, spotkania.id_lekarza, pacjenci.pesel, pacjenci.imie, pacjenci.nazwisko, specjalizacje.nazwa_specjalizacji, lekarze.imie, lekarze.nazwisko, lekarze.nr_pokoju, data_odbycia, data_zapisu, stan FROM spotkania, lekarze, specjalizacje, pacjenci where spotkania.id_lekarza=lekarze.id_lekarza and spotkania.id_specjalizacji=specjalizacje.id_specjalizacji and spotkania.id_osoby=pacjenci.id_pacjenta;
 \end{verbatim}>>.

\paragraph{Tytuł zadania.} Formularz zapisu dla pacjentów.  \subparagraph{Wykonawca.} Łukasz Mamak, Paweł Wilczek. \subparagraph{Realizacja.}
Zadanie polegało na utworzeniu formularza zapisu dla pacjentów  przychodni lekarskiej dzięki któremu przyszły pacjent tworzy swoje konto które następnie pozwala na zapisywanie się na ustalone daty i godziny do konkretnych specjalistów.

 \begin{verbatim}
<html>
<head>
<title>Łukasz Szkaradek, Łukasz Mamak i Paweł Wilczek </title>

<META http-equiv=Content-Language content=pl>
<META http-equiv=Content-Type content="text/html; charset=UTF-8">
<style>
a {
    color: black;
    text-decoration: none;
}
</style>
</head>

<body>
<center>
<a href="index.php">Powrót</a> | <a href="recepcja.php">Zaloguj się</a>
<body style="background-size: 100% 200%;" background="tlo.jpg" bgproperties="fixed"> <br> <br>
<img style="margin:-7px;width: 101%; text-align:center;" src="logo.png" alt="Logo" />
<form action="formularz.php" method="POST"><table border="2">
<tr><td align="center" colspan="2">Formularz</td><tr>
<tr><td>Imię:</td><td><input type="text" name="imie_osoby"></td><tr>
<td>Nazwisko:</td><td><input type="text" name="nazwisko_osoby"></td><tr>
<td>Pesel:</td><td><input type="text" name="pesel"></td><tr>
<td>Numer telefonu:</td><td><input type="text" name="tel"></td><tr>
<td align="center" colspan="2"><input type="submit" value="Wyślij"> </form></td></table>
</body>
<br>
<table border="0" cellspacing="0" cellpadding="0">
  <tr> 
    <td style="background: rgba(42,42,42,0.4);" align="center"><font face="verdana" size="1" color="white">
      &nbsp&nbspCopyright &copy; <b>Łukasz Szkaradek, Łukasz Mamak i Paweł Wilczek&nbsp&nbsp</b></td>
  </tr>
</table>
</center>
</html> 
\end{verbatim}


\subsection{Sprint Review/Demo}\label{Sprint2demo}
Naszym celem była realizacja wcześniej ustalonych i zaplanowanych zadań. Dotyczyły one stworzenia formularza zapisu dla pacjentów oraz aktualizacja bazy danych która polegała na utworzeniu trzech nowych tabel oraz dwóch widoków ułatwiających wyświetlanie danych z tabeli.   Podczas realizacji wcześniej przydzielonych zadań, żaden z członków zespołu nie napotkał na istotne, konieczne do odnotowania problemy. Wszystkie z przewidzianych zadań zostały wykonane terminowo, co za tym idzie przyrost  został osiągnięty.  Z racji tematyki naszego projektu, przyrost, postęp będzie dostępny na stronie internetowej przychodni, w adresie poniżej.\newline
\newline
\textit{Demonstracja przyrostu produktu:}\newline
\newline
\textbf{Link do strony}: http://luki9696.000webhostapp.com/projekt/index.php


\section{Sprint 3}\label{Sprint33} \subsection{Cel}\label{Sprint33cel} Sprint 3 ma na celu stworzenie dwóch podstron. Recepcji, która będzie umożliwiać logowanie wraz z wyświetlaniem opcji dla poszczególnych użytkowników: dyrektora, recepcjonisty i lekarza. Opcje te przekierowywać będą do podstrony zarządzanie. Podstrona zarządzanie wyświetlać będzie tabele i umożliwiać ich edycje i dodawanie, tabele wybierane są poprzez metodę GET ze strony recepcji. \subsection{Sprint Planning/Backlog}\label{Sprint33SPB}

\paragraph{Tytuł zadania.} Strona logowania pracowników. \begin{itemize} \item Estymata: XL. \end{itemize}
\paragraph{Tytuł zadania.} Wyświetlanie, edycja, usuwanie. \begin{itemize} \item Estymata: XXL. \end{itemize}


\subsection{Realizacja}\label{Sprint33realizacja}

\paragraph{Tytuł zadania.} Strona logowania pracowników . \subparagraph{Wykonawca.} Łukasz Mamak. \subparagraph{Realizacja.} Zadanie polegało na stworzeniu strony logowania dla pracowników. Realizacja przebiegła planowo, przyrost został osiągnięty.  Kod programu (środowisko \texttt{verbatim}): \begin{verbatim} ?>
<html>
<head>
<title>Przychodnia lekarska</title>
<META http-equiv=Content-Language content=pl>
<META http-equiv=Content-Type content="text/html; charset=windows-1250">
<style>
a {
    color: black;
    text-decoration: none;
}
</style>
</head>
<body style="background-size: 100% 200%;" background="tlo.jpg" bgproperties="fixed">
<div align="center">
<?php
if($_SESSION['zalogowany']=="ok") {
echo '<a href="recepcja.php?stan=wyloguj">Wyloguj</a> | ';
if (isset($_GET['haslo'])) {
echo '<a href="recepcja.php">Powrót</a> |';
} else {
echo '<a href="recepcja.php?haslo">Zmień hasło</a> |';
}
}
?> <a href="index.php">Strona główna</a><br><br>
<img style="margin:-7px;width: 101%; text-align:center;" src="logo.png" alt="Logo" />
 \end{verbatim}>>.

\paragraph{Tytuł zadania.}Wyświetlanie, edycja, usuwanie.  \subparagraph{Wykonawca.} Łukasz Szkaradek, Paweł Wilczek. \subparagraph{Realizacja.}Realizacja zadania przebiegła pomyślnie, zadanie wymagało wnikliwej analizy i organizacji elementów. W efekcie powstała działająca strona umożliwiająca edycje, dodawanie i wyświetlanie tablic w bazie danych. Wykonanie strony obyło się bez większych problemów. 
\begin{verbatim} 
if($_SESSION['zalogowany']=="ok")
{
switch($_GET['l'])
	{
	    
	case "aktualnosci":
	if ($_SESSION['baza'] == 'dyrektor' or $_SESSION['baza'] == 'recepcja') {
	if(isset($_GET['id_aktualnosci_u'])) {
	$zapytanie_u="DELETE from aktualnosci WHERE id_aktualnosci='".$_GET['id_aktualnosci_u']."'";
	$wykonaj_u = mysqli_query($link, $zapytanie_u);
	header('Location: wyswietl.php?l='.$_GET['l']);
	}
    	if(isset($_POST['id_aktualnosci_e'])) { 
    	$zapytanie_e="UPDATE aktualnosci SET data='".$_POST['data']."', opis='".$_POST['opis']."' WHERE id_aktualnosci='".$_POST['id_aktualnosci_e']."'";
	$wykonaj_e = mysqli_query($link, $zapytanie_e);
	header('Location: wyswietl.php?l='.$_GET['l']);
    	}
    	if(isset($_POST['data_d'])) { 
	if ($_POST['data_d']=="") {
	$zapytanie_d="INSERT into aktualnosci (opis) values('".$_POST['opis']."')";
	} else {
	$zapytanie_d="INSERT into aktualnosci (data, opis) values('".$_POST['data_d']."', '".$_POST['opis']."')";
	}
    	$wykonaj_d=mysqli_query($link,$zapytanie_d);
   	header('Location: wyswietl.php?l='.$_GET['l']);
 }
    
    
	echo '<button onclick="dodaj()">Dodaj</button>&nbsp;<button onclick="edycja()">Edytuj</button><br>';
	echo '<br> Lista aktulaności <br>';
	echo '<table border="1" cellspacing="0" cellpadding="0">';
	echo "<td>ID aktualności</td><td>Data</td><td>Opis</td><td></td><td></td>";
	$zapytanie = "select * from aktualnosci";
	$wykonaj = mysqli_query($link, $zapytanie);
	while($wiersz=mysqli_fetch_assoc($wykonaj)) {
	echo " <tr>
	<td>".$wiersz['id_aktualnosci']."</td>
	<td>".$wiersz['data']."</td>
	<td>".$wiersz['opis']."</td>";
	echo '<td><button onclick="dane(\'' . $wiersz['id_aktualnosci'] . '\',\'' . $wiersz['data'] . '\',\'' . $wiersz['opis'] . '\')">Edytuj</button></td>';
	echo '<td><a href="wyswietl.php?l=aktualnosci&id_aktualnosci_u=' . $wiersz['id_aktualnosci'] . '"/><button>Usuń</button></td>';
	}
	echo '</table><br><div style="height:150px;">';
	
	echo '<table id="edycja" style="display: none" border="1" cellspacing="0" cellpadding="0">';
	echo '<form action="wyswietl.php?l=aktualnosci" method="post">';
	echo '<tr><td>ID aktualności</td><td><input type="text" id="a" name="id_aktualnosci_e"></tr>
	<tr><td>Data</td><td><input type="text" id="b" name="data"><td>
	<tr><td>Opis</td><td><input type="text" id="c" name="opis"><td>';
	echo '</table><input id="p_edycja" style="display: none" type="submit" value="Edytuj">';
	echo '</form><br><br>';
	echo '<table style="display: none" id="dodaj" border="1" cellspacing="0" cellpadding="0">';
	echo '<form action="wyswietl.php?l=aktualnosci" method="post">';
	echo '<tr><td>Data</td><td><input type="text" name="data_d"><td>
	<tr><td>Opis</td><td><input type="text" name="opis"><td>';
	echo '</table><input style="display: none" id="p_dodaj" type="submit" value="Dodaj">';
	echo '</form></div>';
	}
	break;
}
}
     \end{verbatim}


\subsection{Sprint Review/Demo}\label{Sprint33demo} Sprint został wykonany w pełni, nie zostały odnotowane żadne błędy lub komplikacje. Całość w pełni współgra z zadaniami wykonanymi w poprzednich sprintach. Zakończenie tego sprintu jest jednocześnie zakończeniem prac nad całym projektem, gdyż otrzymaliśmy w pełni sprawną i funkcjonalną stronę internetową.
\newline
\textit{Demonstracja przyrostu produktu:}\newline
\newline
\textbf{Link do strony}: http://luki9696.000webhostapp.com/projekt/index.php


\section{Sprint 4}
\label{Sprint22}
\subsection{Cel} Motywem przewodnim sprintu jest przede wszystkim naniesienie niezbędnych poprawek a także udoskonalenie wdrożonych wcześniej rozwiązań, które to sprawią, że zarówno wygląd jak i funkcjonalność zostaną poprawione i poszerzone o niezbędne, a brakujące wcześniej opcje.
\label{Sprint4cel} 

\subsection{Sprint Planning/Backlog}
\label{Sprint4SPB}
\paragraph{Tytuł zadania.} Opcja zmiany hasła użytkowników recepcji. \begin{itemize} \item Estymata: L. \end{itemize}

\paragraph{Tytuł zadania.} Możliwość automatycznego wpisywania danych logowania pacjenta po IP. \begin{itemize} \item Estymata: L. \end{itemize}

\paragraph{Tytuł zadania.} Poprawa formularza rejestracji pacjenta.  \begin{itemize} \item Estymata: S. \end{itemize}



\subsection{Realizacja}
\label{Realizacja4}
\paragraph{Tytuł zadania.}  Opcja zmiany hasła użytkowników recepcji. \subparagraph{Wykonawca.} Łukasz Szkaradek, Łukasz Mamak \subparagraph{Realizacja.}  
\newline
\newline
Kod programu (środowisko \texttt{verbatim}): \begin{verbatim} \end{verbatim}

\paragraph{Tytuł zadania.}  Możliwość automatycznego wpisywania danych logowania pacjenta po IP. \subparagraph{Wykonawca.} Paweł Wilczek, Łukasz Mamak \subparagraph{Realizacja.}  
\newline
\newline
Kod programu (środowisko \texttt{verbatim}): \begin{verbatim} \end{verbatim}


\paragraph{Tytuł zadania.}  Poprawa formularza rejestracji pacjenta.  \subparagraph{Wykonawca.} Łukasz Szkaradek\subparagraph{Realizacja.}  
\newline
\newline
Kod programu (środowisko \texttt{verbatim}): \begin{verbatim} \end{verbatim}






\subsection{Sprint Review/Demo}\label{Sprint4demo}
\newline
\newline
\textit{Demonstracja przyrostu produktu:}\newline
\newline
\textbf{Link do strony}: http://luki9696.000webhostapp.com/projekt/index.php


\begin{thebibliography}{9}

\bibitem{Cov} S. R. Covey, {\em 7 nawyków skutecznego działania}, Rebis, Poznań, 2007.

\bibitem{Oet} Tobias Oetiker i wsp., Nie za krótkie wprowadzenie do systemu \LaTeX \ $2_\varepsilon$, \url{ftp://ftp.gust.org.pl/TeX/info/lshort/polish/lshort2e.pdf}

\bibitem{SchSut} K. Schwaber, J. Sutherland, {\em Scrum Guide}, \url{http://www.scrumguides.org/}, 2016.

\bibitem{apr} \url{https://agilepainrelief.com/notesfromatooluser/tag/scrum-by-example}

\bibitem{us} \url{https://www.tutorialspoint.com/scrum/scrum_user_stories.htm}

\end{thebibliography}

\end{document}

% ----------------------------------------------------------------
